%%=============================================================================
%% Samenvatting
%%=============================================================================

% TODO: De "abstract" of samenvatting is een kernachtige (~ 1 blz. voor een
% thesis) synthese van het document.
%
% Deze aspecten moeten zeker aan bod komen:
% - Context: waarom is dit werk belangrijk?
% - Nood: waarom moest dit onderzocht worden?
% - Taak: wat heb je precies gedaan?
% - Object: wat staat in dit document geschreven?
% - Resultaat: wat was het resultaat?
% - Conclusie: wat is/zijn de belangrijkste conclusie(s)?
% - Perspectief: blijven er nog vragen open die in de toekomst nog kunnen
%    onderzocht worden? Wat is een mogelijk vervolg voor jouw onderzoek?
%
% LET OP! Een samenvatting is GEEN voorwoord!

%%---------- Nederlandse samenvatting -----------------------------------------
%
% TODO: Als je je bachelorproef in het Engels schrijft, moet je eerst een
% Nederlandse samenvatting invoegen. Haal daarvoor onderstaande code uit
% commentaar.
% Wie zijn bachelorproef in het Nederlands schrijft, kan dit negeren, de inhoud
% wordt niet in het document ingevoegd.

\IfLanguageName{english}{%
\selectlanguage{dutch}
\chapter*{Samenvatting}

Quantum computing staat op het punt van de volgende grote ingrijpende technologie te worden.
Toch zijn het vooral de mensen die met quantum computers bezig zijn, die weten welke belangrijke rol het kan spelen in de toekomst.
Dit is de reden van het schrijven van deze studie: kennis geven over quantum computing en zijn voordelen aan het grotere publiek.
Deze studie moet de voordelen die quantum computing kan hebben uitleggen. Zo moet het CEO's en CIO's helpen met het begrijpen hoe een quantum computer werkt zodat zij de voordelen voor hun bedrijf kunnen zien.
Om te starten legt deze studie de basisbegrippen uit van quantum computers. Met de kennis rond quantum computers kan er gekeken worden naar algoritmes die beter zijn op een quantum computer dan op een klassieke computer.
Het algoritme dat onderzocht werd is Shor's algoritme. Dit is de quantum tegenhanger van de getallenlichamenzeef. Beide zijn algoritmes die grote nummers (N > $10^{100}$) gaan ontbinden in factoren.
Wat hieruit bleek is dat het aantal cijfers in het nummer een grote impact had op het klassieke algoritme. Een toevoeging van slechts 20 cijfers in het nummer kan de execution tijd doen oplopen van een klein uur tot wel 13 uur.
Shor's algoritme werkt momenteel enkel op kleine cijfers (het grootste dat een resultaat gaf was 119) en de zeef werkte enkel op grootte cijfers (N > $10^{60}$), toch kan er geconcludeerd worden uit de tijd en geheugen complexiteiten, dat een quantum algoritme de tijd en het geheugen dat nodig is voor het bekomen van een oplossing, kan verbeteren.
Momenteel staat de technologie van quantum computing nog niet ver genoeg om gebruikt te worden in gewone bedrijven, zoals bleek uit het quantum gedeelte van de studie.
De hamvraag blijft dan: hoelang zal het duren voor deze technologie wel implementeerbaar is in bedrijven?
\selectlanguage{english}
}{}

%%---------- Samenvatting -----------------------------------------------------
% De samenvatting in de hoofdtaal van het document

\chapter*{\IfLanguageName{dutch}{Samenvatting}{Abstract}}

Quantum computing is on the verge of becoming the next big thing. Yet mostly only people who work with quantum computers know what they can bring to the table.
This study should help understanding their advantages.
Its purpose is to educate CEOs and CIOs about quantum computing so that when quantum computing starts getting integrated into the businesses, they have an understanding of what it can mean for them.
The study looked into the effects that quantum computing could have on the industry and looked into its implementations in different fields.
The effects were measured by comparing two algorithms with the same task, factoring a large number (N > $10^{100}$). One of them, the number field sieve, running on a classical computer and the other, Shor's algorithm, on a quantum computer.
Even though Shor's algorithm only worked on small numbers (the highest number taht gave results was 119) and that NFS only worked on large numbers (N > $10^{60}$), there can be concluded that a quantum solution for a problem can both improve the time and the space (this is the memory) needed to find an outcome.
But at this moment quantum computers are not sophisticated enough to handle big tasks. As seen in the study, Shor's algorithm could only factor rather small numbers. Since these numbers are not used in applications today, Shor won't have any effect.
The biggest question there still is: how long will it take to improves this technology so that is has an impact on the industry?
