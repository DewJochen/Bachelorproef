%%=============================================================================
%% Conclusie
%%=============================================================================

\chapter{Conclusie}
\label{ch:conclusie}

% TODO: Trek een duidelijke conclusie, in de vorm van een antwoord op de
% onderzoeksvra(a)g(en). Wat was jouw bijdrage aan het onderzoeksdomein en
% hoe biedt dit meerwaarde aan het vakgebied/doelgroep? 
% Reflecteer kritisch over het resultaat. In Engelse teksten wordt deze sectie
% ``Discussion'' genoemd. Had je deze uitkomst verwacht? Zijn er zaken die nog
% niet duidelijk zijn?
% Heeft het onderzoek geleid tot nieuwe vragen die uitnodigen tot verder 
%onderzoek?

This study tried to find an answer on what the effect of quantum computing could be on the industry.

To start a comparison between classic and quantum computing was done. This came to the conclusion that by using quantum mechanics significant improvements could be reached.
The time and space complexities both could be lowered, which can help in solving problems that could not be solved before in a meaningful time frame.
As can be seen the effect of the input data in an algorithm can drasticly change the runtime of it. For example, adding only 20 digits to a number to factor can change the runtime from a little bit under an hour to 13 hours.
And eventhough it's not visible from the experiment itself, this does not happens as extreme on a quantum computer as on a classical one.

At the momement of writing the industries that could benefit most of quantum computing are the fields where research is done. The technology itself is to expensive and not advanced enough to start using in businesscompanies.
When the more advanced quantum computers are made, other industries, like financial institutions, chemistry companies and such can also benifit from this.

