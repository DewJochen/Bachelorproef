%%=============================================================================
%% Inleiding
%%=============================================================================

\chapter{\IfLanguageName{dutch}{Inleiding}{Introduction}}
\label{ch:inleiding}

There is no denying that everything is changing rapidly. For most people it is so much, that keeping up is really difficult.
Look at the phone industry: every year a brand comes out with a new model. Yet the differences with the older version are so minor that most people don't even notice them.
A better camera, 5G, or some other features few people ever use. So, does this really have a benefit? Or is it just a selling trick?
That is what this study is trying to figure out, only it is not about phones but about quantum computers.

\section{\IfLanguageName{dutch}{Probleemstelling}{Problem Statement}}
\label{sec:probleemstelling}
Quantum computing is next big thing, or is it? There have been researches on quantum computing for forty-some years.
Now that the bigger companies, like IBM or Google, are also looking into it, it gets more and more attention. 
But could it have a real impact on the world? And if so, what are the effects it will have on the industry?
This knowledge can be helpful for managers who want to start using quantum computing in their business, so they are aware of the benefits and even the risks for the big investment they are making.
It also may help researchers in improving their quantum computers.

\section{\IfLanguageName{dutch}{Onderzoeksvraag}{Research questions}}
\label{sec:onderzoeksvraag}

The function of this study is to gain insight on what effects quantum computers have be on the industry.
Not only the benefits, but also the drawbacks should be investigated. In order to do so, a comparison is made between a 'normal' algorithm and a quantum algorithm.
In the comparison the time and space complexity of both algorithms is investigated in section \ref*{subsec:Complexities}.
In completion of the study these questions will be answered:

\begin{itemize}
  \item What is the difference between quantum computing and classic computing? The comparison should be very basic but for some specific items it will go a bit more in depth, such as the time-space complexities.
  \item Does the number of inputs on the algorithms have a significant effect on the time it takes to run the algorithm?
  \item What are the most important business processes that can be improved with quantum computing?
  \item What is the effect of quantum computing on these processes?
\end{itemize}

\section{\IfLanguageName{dutch}{Onderzoeksdoelstelling}{Research objective}}
\label{sec:onderzoeksdoelstelling}

Based on the research question the following list should definitely be in the research:

\begin{itemize}
  \item A clear explanation on what quantum computing is.
  \item A study on differences between classic and quantum computing.
  \item A classic algorithm that can be translated into a quantum algorithm.
  \item A study on the time-space complexity of both algorithms
  \item A survey to find the most important business processes.
  \item A study on the effects of quantum computing on the found processes.
\end{itemize}

\section{\IfLanguageName{dutch}{Opzet van deze bachelorproef}{Structure of this bachelor thesis}}
\label{sec:opzet-bachelorproef}

% Het is gebruikelijk aan het einde van de inleiding een overzicht te
% geven van de opbouw van de rest van de tekst. Deze sectie bevat al een aanzet
% die je kan aanvullen/aanpassen in functie van je eigen tekst.
The rest of the thesis will be as followed:

Chapter~\ref{ch:stand-van-zaken} will give an answer to what quantum computing is, what algorithms will be used and a survey will be constructed. \\
In chapter~\ref{ch:methodologie} the methodology will be given. How will the study be conducted to be sure the answers are clear and correct? \\
Chapter~\ref{ch:onderzoek} will handle the experiments itself.\\
Finally, in the last chapter~\ref{ch:conclusie} the conclusion of this study will be given. What are the effects on the industry and what can we do with these?
