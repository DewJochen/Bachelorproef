%%=============================================================================
%% Conclusie
%%=============================================================================

\chapter{Conclusie}
\label{ch:conclusie}

% TODO: Trek een duidelijke conclusie, in de vorm van een antwoord op de
% onderzoeksvra(a)g(en). Wat was jouw bijdrage aan het onderzoeksdomein en
% hoe biedt dit meerwaarde aan het vakgebied/doelgroep? 
% Reflecteer kritisch over het resultaat. In Engelse teksten wordt deze sectie
% ``Discussion'' genoemd. Had je deze uitkomst verwacht? Zijn er zaken die nog
% niet duidelijk zijn?
% Heeft het onderzoek geleid tot nieuwe vragen die uitnodigen tot verder 
%onderzoek?

This study tried to find an answer on what the effect of quantum computing could be on the industry.

To start a comparison between classic and quantum computing was done. And like expected, this came to the conclusion that by using quantum mechanics significant improvements could be reached.
Improvements on the time and space complexities could help in solving problems that just took too long on a classical computer. This does not mean that a problem that had no solution on a classical computer, now has one.
As can be seen the effect of the input data in an algorithm can drasticly change the runtime of it. For example, adding only 20 digits to a number to factor can change the runtime from a little bit under an hour to 13 hours on a classical computer.
And even though it's not visible from the experiment itself, this does not happen as extreme on a quantum computer as on a classical one.
In a quantum computer some calculations can be run at the same time, which is not possible in a classical computer. So, it was expected that a quantum computer would speed up the running time of algorithms.

The most important business processes that could improve from quantum computing are processes with a lot of data. This is because of Grover's algorithm which can find a value in the data signifantly faster than a normal search could do.
At the momement of writing the industries that could benefit most of quantum computing are the fields where research is done. The technology itself is to expensive and not advanced enough to start using in companies.
When the more advanced quantum computers are made, other industries, like financial institutions, chemistry companies, etc. can also benifit from this.

It is uncertain when this technology will be mature enough to start using. Experts believe that the timeframe for a system that is advanced enough is around 2026-2030 \autocite{timeframe}.
Even when quantum computers start getting integrated into companies, the changes won't be seen overnight. It will probably take a long time untill everything is sorted out, for optimal use.

