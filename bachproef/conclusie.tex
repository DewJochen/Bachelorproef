%%=============================================================================
%% Conclusie
%%=============================================================================

\chapter{Conclusie}
\label{ch:conclusie}

% TODO: Trek een duidelijke conclusie, in de vorm van een antwoord op de
% onderzoeksvra(a)g(en). Wat was jouw bijdrage aan het onderzoeksdomein en
% hoe biedt dit meerwaarde aan het vakgebied/doelgroep? 
% Reflecteer kritisch over het resultaat. In Engelse teksten wordt deze sectie
% ``Discussion'' genoemd. Had je deze uitkomst verwacht? Zijn er zaken die nog
% niet duidelijk zijn?
% Heeft het onderzoek geleid tot nieuwe vragen die uitnodigen tot verder 
%onderzoek?

This study tried to find an answer on what the effect of quantum computing could be on the industry.

To start a comparison between classic and quantum computing was done. This came to the conclusion that by using quantum mechanics significant improvements could be reached.
The time and space complexities both could be lowered, which can help in solving problems that could not be solved before in a meaningful time frame.
Another conclusion can be made that for some algorithms the amount of data used can drasticly change the effectiveness of the algorithm. For example the RSA-120, which has a 120 number, could be factored classically in 26 hours of CPU-time.
Only adding 35 digits increased the time from 26 hours to 83 days. On a quantum computer this difference would be notably smaller.

At the momement of writing the industries that could benefit most of quantum computing is where al lot of research is done.

