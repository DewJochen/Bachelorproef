%%=============================================================================
%% Samenvatting
%%=============================================================================

% TODO: De "abstract" of samenvatting is een kernachtige (~ 1 blz. voor een
% thesis) synthese van het document.
%
% Deze aspecten moeten zeker aan bod komen:
% - Context: waarom is dit werk belangrijk?
% - Nood: waarom moest dit onderzocht worden?
% - Taak: wat heb je precies gedaan?
% - Object: wat staat in dit document geschreven?
% - Resultaat: wat was het resultaat?
% - Conclusie: wat is/zijn de belangrijkste conclusie(s)?
% - Perspectief: blijven er nog vragen open die in de toekomst nog kunnen
%    onderzocht worden? Wat is een mogelijk vervolg voor jouw onderzoek?
%
% LET OP! Een samenvatting is GEEN voorwoord!

%%---------- Nederlandse samenvatting -----------------------------------------
%
% TODO: Als je je bachelorproef in het Engels schrijft, moet je eerst een
% Nederlandse samenvatting invoegen. Haal daarvoor onderstaande code uit
% commentaar.
% Wie zijn bachelorproef in het Nederlands schrijft, kan dit negeren, de inhoud
% wordt niet in het document ingevoegd.

\IfLanguageName{english}{%
\selectlanguage{dutch}
\chapter*{Samenvatting}

\selectlanguage{english}


%%---------- Samenvatting -----------------------------------------------------
% De samenvatting in de hoofdtaal van het document

\chapter*{\IfLanguageName{dutch}{Samenvatting}{Abstract}}

Since quantum computing is on the verge of becoming the next big thing and mostly only people who work with them know what what they can bring to the table, this study shoudl help with understanding their advantages.
It's purpose is to educate CEO's and CIO's about quantum computing so that when it integrates into the businesses they have a understanding of what it can mean for them.
The study looked into the effects that quantum computing could have on the industry and looked into its implementaions in different fields.
The effects were measured by comparing two algorithms with the same task, factoring a large number(N > $10^100$), one of them, the number field sieve, running on a classical computer and the other, Shor's algorithm, on a quantum computer.
Even though Shor's algorithm only worked on small numbers (the max was 119) and that NFS only worked on large numbers (N > $10^60$), there can be concluded that a quantum solution for a problem can both improve the time needed as the space needed to find this outcome.

