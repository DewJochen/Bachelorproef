%%=============================================================================
%% Onderzoek
%%=============================================================================

\chapter{\IfLanguageName{dutch}{Onderzoek}{Study}}
\label{ch:onderzoek}



In this study there will be a comparison between the GNFS algorithm on a classical computer and Shor's algorithm on a quantum computer.
In section \ref{subsec:Other algorithms} discussing some other algorithms that can show benefits of quantum computing in the industry. To conclude there will be looked into the prime industries it could be used in.

\section{The comparison}
\subsection{General Number Field Sieve}
\subsection{Shor's algorithm}
\subsubsection{Gates}
\label{gates}
For building circuits, this study will be using IBM's quantum composer. The composer can be used in two differnt ways: drag and drop or writing code.
As mentioned in section \ref{sec:Composer} writing code in the composer is done with OpenQASM2.0 and the drag and drop uses lines that represent qubits on which gates or operators can be placed.
Before making the algorithm, first the gates have to be illustrated. All the information of the operations can be found on the webside of IBM \footnote{$https://quantum-computing.ibm.com/composer/docs/iqx/operations_glossary$} on quantum computing as well as in \textcite{Hidary_2019} section 3.1 Quantum operators.
There are 5 types of gates: classicals, quantums, phases, non-unitaries and the Hadamard. Each of these are reversible: when the same operation is used twice on a qubit, it's state will be the same as the starting state \autocite{reversible_gates, revgates}.

The first gates that will be discussed are the classical gates, also known as Pauli gates. In this group 4 gates can be found: the NOT gate, the CNOT gate, the Toffoli gate and the SWAP gate.
\section{Other algorithms}
\label{Other algorithms}
\section{The use of quantum computing}
